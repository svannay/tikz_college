% !TeX encoding = UTF-8
% !TeX program = xelatex
% !TeX spellcheck = fr
\documentclass[10pt,a4paper]{report}

\usepackage[informatique]{styles/preambule_college}
\usepackage{styles/preambule_personnalisation}


\title{Fiches pratiques}
\author{S. Vannay}
\date{12.09.2022}

\begin{document}





\chapterFormat
\chapter*{Figures 2D de GeoGebra au QCM Moodle}





\section*{Workflow et fichiers utilisés}



\subsection*{Avec GeoGebra}

\begin{enumerate}
	\item Le dessin de la figure se fait en GeoGebra dans le répertoire \incmd{images_qcm}.
	\item Dans GeoGebra, on procède de manière similaire à celle des figures pour les supports de cours avec les nuances suivantes :
		\begin{itemize}
			\item \incmd{exportHight = 12}, \incmd{exportWidth = 6.5} (pour mieux correspondre à l'écran d'un téléphone); \newline
				\textbf{Rem : } avec les points \incmd{Export}\textsubscript{1} et \incmd{Export}\textsubscript{2}, il n'y a pas besoin d'adapter la fenêtre \incmd{Graphique} de GeoGebra pour délimiter la zone à exporter. Il suffit que ces deux points soient dans la zone visible.
			\item \incmd{Fichier} \textrightarrow \ \incmd{Exporter} \textrightarrow \ \incmd{Graphique en tant qu}'\incmd{image (png, eps)...}
			\item Laisser en format png, au besoin régler l'échelle.
			\item Au besoin monter la résolution en 300 dpi.
			\item Décocher \incmd{Transparent} pour avoir l'image sur fond blanc (pour garantir le contraste et les couleurs) et sauvegarder.
		\end{itemize}
	\item Dans le fichier \incmd{.tex}, importer l'image avec
		\begin{center}
			\inlatex{\includegraphics[width=6.5cm]{...}}
		\end{center}
\end{enumerate}



\subsection*{Avec des photos}

Ne pas exagérer la taille des fichiers (env. 300kB au maximum) et compiler avec Lua pour éviter un problème de mémoire dynamique lors de la conversion des images.


\end{document}